\documentclass{article}
\usepackage{amsmath}
\usepackage{enumitem}
\title{Stats1}
\author{Jakub Slowinski}
\begin{document}

\begin{enumerate}
  \item 
  \begin{enumerate}[label=(\alph*)]
    \item There are 10 possible letters.
    \[10 * 9 * 8 * 7 * 6 * 5 * 4 * 3 * 2 * 1\]
    \[10! = 3628800\]
    \item The 2 letters have to be next to each other, EF counts as one item and there are 9 items now that can be arranged in any order. E and F can be order of: EF or FE hence 2!, which is just 2.
    \[9! * 2! = 725760\]
    \item There are 6 letters so 6! different combinations dividided by the amount of combination of Ns(2!) and As(3!), there is only 1 B so it doesnt matter. 
    \[ \frac{6!}{3! * 2!} = 60 \]
    \item 5 possible letters, choose 3.
    \[ \binom{5}{3} = 10 \]
  \end{enumerate}
  \item
  \begin{enumerate}[label=(\alph*)]
  
    \item 6 sided dice rolled 4 times.
    \[ 6^4 = 1296 \]
    \item 4 dice, 2 of them roll a 3 and 2 roll anything other than a 3.
    \[ \binom{4}{2} * 5^2 = 150 \]
    
    \item There are 3 possible outcomes which you have to add together, a)you can have 2 3s(4C2) and the rest of the dice can't be 3 so 5 possibilities on each, b)3 3s(4C3) and another dice with 5 possibilities, c) 4 4s(4C4).
    \[ \binom{4}{2} * 5^2 = 150\]
    \[ \binom{4}{3} * 5   = 20\]
    \[ \binom{4}{4}       = 1 \]
    \[ 150+20+1 = 171 \]
    
  \end{enumerate}
  \item
  \begin{enumerate}[label=(\alph*)]
  
    \item 8 different cards so 8!, but there are 4 suits containing 2 aces each.
    \[ \frac{8!}{2!*2!*2!*2!} = 2520 \]
    
    \item 8 cards = 4 distinct cards, choose 2.
    \[ \binom{4}{2} = 6 \]
    \item The good suits are half of the total suit amount. Since you still have 2 cards you can half the answer from Q3b to receive 3 as the amount of ways to get good cards.
  \end{enumerate}
\end{enumerate}
\end{document}